% mensaje: 
Feature selection on homogeneous categorical data set containing thousands of variables needs careful management of relations and dependencies based on sole values. Any semantical information such as gene or proteins interactions should be taken into account as a possible support entity.
\\

% mensaje:
Regarding the accuracy of results, the available parameters allow for a more strict or loose constraint setting. Severe thresholds like Swap score = 1 and Quasi-constant leave-out index = 0.3 (or higher, closer to 0.5) would result in a more solid result. Nevertheless, any nodes or relations fulfilled the requirements, so no graph could be built from it. Based on this fact, we're forced to establish a trade-off among both constraints, according to our needs.
\\

% mensaje:
In our specific case, the related nodes hold a semantic relation too\footnote{The variant names are hidden for confidentiality purposes, its veracity can be checked by requesting the real data access (See \ref{availability})}. 65\% of the connections relate SNPs which are (1) in the same chromosome, and (2) their loci base-distance is within 1-10k. Considering the grouping approaches we have worked with (rivalry, correlation, 1k loci, 10k loci and genes) which were proposed with an experimental objective, there might be some support from the swapper's side on it. In any case, SNPs located in different chromosomes or faraway within the same chromosome still hold 35\% of the cases, which should be further analysed from a biological point of view to validate to which extent this result is accurate.
\\

As seen in the results, data type heavily impact on swapper's performance when aiming for rivalry detection. Continuous values, due to their variant nature, allow for a better and more accurate discerning than discrete values or categorical labels. Therefore, the burden of the approach is not entirely due to the type of data side, which fundamentally introduces a a challenge because of the presence of quasi constant variables, but also due to the potential a kind of data holds of having variability, e.g. gene expression datasets.
This flaw can be corrected by applying proper corrections in the preprocessing step or, like in this work, once they become discoverable, as a post-processing procedure.