\documentclass{jot}
\usepackage[utf8]{inputenc}
\usepackage[T1]{fontenc}
\usepackage[english]{babel}
\usepackage{subfiles}
\usepackage{titlesec}
\usepackage{multicol}
\usepackage[titles]{tocloft}
\usepackage{adjustbox}
\usepackage{multirow}
\usepackage{tabularx}
\usepackage{hyperref}
\usepackage{cleveref}
\usepackage{placeins}
\usepackage{flafter}

\crefformat{figure}{#2{\color{blue}#1}#3}
\Crefformat{figure}{#2{\color{blue}#1}#3}

\crefformat{equation}{#2{\bfseries\color{violet}#1}#3}
\Crefformat{equation}{#2{\itshape\color{violet}#1}#3}

\usepackage{subcaption}
\usepackage{mathtools}% http://ctan.org/pkg/mathtools

\newcolumntype{Y}{>{\raggedright\arraybackslash}X}

\setcounter{secnumdepth}{4}

\titleformat{\paragraph}
{\normalfont\normalsize\bfseries}{\theparagraph}{1em}{}
\titlespacing*{\paragraph}
{0pt}{3.25ex plus 1ex minus .2ex}{1.5ex plus .2ex}



\usepackage{microtype} % optional, for aesthetics
\usepackage{tabularx} % nice to have
\usepackage{booktabs} % necessary for style
% \usepackage{graphicx}
% \graphicspath{{./figures/}}
% \usepackage{listings}
% \lstset{...}

% \newcommand\code[1]{\texttt{#1}}
% \let\file\code


%%% Article metadata
%\title{Implementation, benchmarking and application of functional analysis workflows of omics data}

\runningtitle{Feature Selection on Medical Data}

\author[affiliation=orgname, nowrap] % , photo=FILE]
    {Raul I. Coroban}
    {is ...
    Contact him at \email{r.coroban@student.vu.nl}}

%\affiliation{orgname}{ORGANISATION}

\runningauthor{R.I. Coroban}

%\jotdetails{
%    volume=V,
%    number=N,
%    articleno=M,
%    year=2011,
%    doisuffix=jot.201Y.VV.N.aN,
%    license=ccbynd % choose from ccby, ccbynd,ccbyncnd
%}


\begin{document}

\begin{abstract}
This project aims to study the limits of ensemble feature selection varying the amount of variables, observations, inducers and any other hyper-parameter of the algorithm. Relevant variables identification is performed via an ensemble procedure developed by the ITI based on resampling without replacement. In order to detect redundant variables we perform maximum-relevance minimum-redundancy (MRMR) approaches, plus a novel algorithm which individually swaps each feature set item by an ensemble top-picked variable (weak redundancy detection). The SNP dataset has been collected on diabetes disease (~1:158 samples to variables relation). From machine learning point of view, the method largely reduces the curse of dimensionality. From a medical point of view, it should help to focus only on genetical variables related with the disease, making laboratory validation more affordable.
\end{abstract}

\keywords{feature selection; data analysis; diabetes.}

\section{Background}
\label{section:background}
\subfile{1.background.tex}

\section{Methods}
\label{section:methods}
\subfile{2.methods.tex}

\section{Results}
\label{section:results}
\subfile{3.results.tex}

\section{Discussion}
\label{section:discussion}
\subfile{4.discussion.tex}

\section{Conclusions}
\label{section:conclusions}
\subfile{5.conclusions.tex}

\section{Acknowledgements}
\subfile{6.acknowledgements.tex}

\section{Availability}
\subfile{7.availability.tex}

\bibliographystyle{abbrv}
\bibliography{references.bib}

%\abouttheauthors

%\begin{acknowledgments}
%ACKS
%\end{acknowledgments}

\newpage
\appendix
\section{Supplementary Material}
\subfile{supplementary.tex}


\end{document}
